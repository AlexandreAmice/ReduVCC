%%
%% This is file `sample-sigconf.tex',
%% generated with the docstrip utility.
%%
%% The original source files were:
%%
%% samples.dtx  (with options: `sigconf')
%% 
%% IMPORTANT NOTICE:
%% 
%% For the copyright see the source file.
%% 
%% Any modified versions of this file must be renamed
%% with new filenames distinct from sample-sigconf.tex.
%% 
%% For distribution of the original source see the terms
%% for copying and modification in the file samples.dtx.
%% 
%% This generated file may be distributed as long as the
%% original source files, as listed above, are part of the
%% same distribution. (The sources need not necessarily be
%% in the same archive or directory.)
%%
%% The first command in your LaTeX source must be the \documentclass command.
\documentclass[sigconf]{acmart}

\newif\ifFull
\Fullfalse

\newif\ifDoubleBlind
\DoubleBlindtrue
%\DoubleBlindfalse

\newcommand{\Id}[1]{{\texttt{#1}}}
%%%% Changes Display 

%\usepackage[final]{changes}
\usepackage{changes}
\definechangesauthor[name={Darren Strash}, color=blue]{DS}
\definechangesauthor[name={Louise Thompson}, color=green]{LT}

\usepackage{textcomp}
\usepackage{amsmath}
\let\proof\relax
\let\endproof\relax
\usepackage{amsthm}
\usepackage{amsfonts}
\usepackage{algorithm}
\usepackage[noend]{algpseudocode}
\usepackage{mathtools}
\usepackage{wasysym}
\usepackage{array}
\usepackage{subcaption,siunitx,booktabs}
\usepackage{numprint}
\usepackage{hyperref}
\npdecimalsign{.} % we want . not , in numbers

\usepackage{color, colortbl}
\definecolor{Gray}{gray}{0.9}
\definecolor{darkblue}{RGB}{0, 0, 255}

\DeclarePairedDelimiter{\ceil}{\lceil}{\rceil}
% fixing lines ending on one word etc...
\setlength\parfillskip{0pt plus .4\textwidth}
\setlength\emergencystretch{.1\textwidth}
\clubpenalty10000
\widowpenalty10000
\displaywidowpenalty=10000

\usepackage{xspace}
\newcommand{\Is}       {:=}
\newcommand{\setGilt}[2]{\left\{ #1\sodass #2\right\}}
\newcommand{\sodass}{\,:\,}
\newcommand{\set}[1]{\left\{ #1\right\}}
\newcommand{\todoupdate}[1]{{\color{red}#1}}
\newcommand{\gilt}{:}
\newcommand{\pic}{}
\newcommand{\erdos}{Erd{\H o}s-R{\'e}nyi}

\def\MdR{\ensuremath{\mathbb{R}}}
\newcommand{\ie}{i.\,e.,\xspace}
\newcommand{\eg}{e.\,g.,\xspace}
\newcommand{\etal}{et~al.\xspace}

\bibliographystyle{plainurl}

\usepackage{xspace}
\newcommand{\external}[0]{external\xspace}

\usepackage{enumerate}

%% Rights management information.  This information is sent to you
%% when you complete the rights form.  These commands have SAMPLE
%% values in them; it is your responsibility as an author to replace
%% the commands and values with those provided to you when you
%% complete the rights form.
\setcopyright{acmcopyright}
\copyrightyear{2019}
\acmYear{2019}
\acmDOI{}

%% These commands are for a PROCEEDINGS abstract or paper.
\acmConference[SIGMOD '20]{SIGMOD '20: ACM International Conference on Management of Data}{Jun 14--19, 2020}{Portland Oregon}
\acmBooktitle{SIGMOD '20: ACM International Conference on Management of Data,
  June 14--19, 2020, Portland, OR}
\acmPrice{}
\acmISBN{}


%%
%% Submission ID.
%% Use this when submitting an article to a sponsored event. You'll
%% receive a unique submission ID from the organizers
%% of the event, and this ID should be used as the parameter to this command.
%%\acmSubmissionID{123-A56-BU3}

%%
%% The majority of ACM publications use numbered citations and
%% references.  The command \citestyle{authoryear} switches to the
%% "author year" style.
%%
%% If you are preparing content for an event
%% sponsored by ACM SIGGRAPH, you must use the "author year" style of
%% citations and references.
%% Uncommenting
%% the next command will enable that style.
%%\citestyle{acmauthoryear}

%%
%% end of the preamble, start of the body of the document source.
\begin{document}

%%
%% The "title" command has an optional parameter,
%% allowing the author to define a "short title" to be used in page headers.
\ifDoubleBlind
\title{Kernelization for the Minimum Vertex Clique Cover Problem}
\fi

\date{}

\pagestyle{plain}

\ifDoubleBlind
\author{}

%\authorrunning{Anonymous}
%\Copyright{Anonymous}
\else
\author{Darren Strash}
\authornote{Both authors contributed equally to this research.}
\email{dstrash@hamilton.edu}
\orcid{0000-0001-7095-8749}
\authornotemark[1]
\affiliation{%
  \institution{Hamilton College}
  \streetaddress{198 College Hill Road}
  \city{Clinton}
  \state{New York}
  \postcode{13323}
}
\author{Louise Thompson}
\authornotemark[1]
\fi{}%

%%
%% By default, the full list of authors will be used in the page
%% headers. Often, this list is too long, and will overlap
%% other information printed in the page headers. This command allows
%% the author to define a more concise list
%% of authors' names for this purpose.
\renewcommand{\shortauthors}{Strash and Thompson}

%%
%% The abstract is a short summary of the work to be presented in the
%% article.
\begin{abstract}
\end{abstract}

\maketitle

\begin{CCSXML}
\end{CCSXML}

\ccsdesc[500]{Computer systems organization~Embedded systems}
\ccsdesc[300]{Computer systems organization~Redundancy}
\ccsdesc{Computer systems organization~Robotics}
\ccsdesc[100]{Networks~Network reliability}

\keywords{minimum clique cover, data reductions, kernelization, fixed-parameter tractability}

\section{Introduction}
\label{sec:Introduction}

\subsection{Our Results.}

\section{Preliminaries}
\label{sec:Preliminaries}
\subsection{Related Work.}
\label{sec:RelatedWork}

\section{New Data Reduction Rules for Minimum Vertex Clique Cover}
\section{Implementation Details}
\section{Experiments}
\subsection{Experimental Setup.}
\subsection{Data Sets.}
\subsection{Exactly Solved Instances}
\subsection{Speeding up with Heuristic Search}
\subsection{Checking the Quality of Independent Sets}

\bibliographystyle{ACM-Reference-Format}
\bibliography{references}
\end{document}

\endinput
%%
%% End of file `sample-sigconf.tex'.
